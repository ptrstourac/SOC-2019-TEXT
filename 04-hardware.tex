\chapter{Hardware}
Jak již bylo zmíněno, ProtoPlant je modulární systém.
Jádrem celého systému je řídící jednotka. 
K~ní jsou připojeny periferie, jako senzory, čerpadla, atd.

\section{Napájení}
Systém je rozdělen na celkem 2 napájecí okruhy.
Zdrojem el. proudu je zásuvka.
Do ní se s pomocí klasické vidlicové zástrčky připojí měniče první a druhé větve.
První větev s napětím 12~V je používaná k~napájení aktuátorů.
Druhá, 5-ti~voltová větev napájí řídící elektroniku a senzory.

%\fxnote*[author=MB]{místo 3.3 patří 3,3 V}{obrázek}

\section{Řídící jednotka}
Jádrem celého systému je tzv. řídící jednotka.
Vnější obal tvoří vodotěsná průmyslová krabice z~ABS plastu s~krytím IP65.
Ta v sobě ukrývá řídící elektroniku a~invertor polarity (viz. níže).
Na krajích řídící jednotky se nachází průchodky s~kabelovými vývody pro připojení ostatního hardwaru.
Na přední straně se nachází znakový lcd display (20x4) a několik průmyslových tlačítek sloužících k ručnímu ovládání zavlažování a oken.

\subsection{ESP32 DevKit C}
Celý systém je řízen touto deskou, která je osazena mikroprocesorem ESP32. 
Celá deska má celkem 37 pinů, z toho jeden napájecí a dva zemnící. 
16 pinů lze použít jako analogové.

Důvod, proč jsem se rozhodl své řešení založit na ESP32 DevKitu C je jednoduchý.
Deska je univerzální, má poměrně dobrý výkon a sama o sobě podporuje WiFi, čímž je možno systém vylepšit o možnost ovládání přes internet, když jsme například na dovolené. 

Samotná deska je určena pro napájení 5~V, je tedy připojena k druhé napájecí větvi.

\subsection{Invertor polarity pro pohon oken}
Toto zařízení slouží k měnění polarity lineárních aktuátorů používaných pro otevírání a zavírání oken.
Sestává se ze tří spínacích relé, z nichž jedno řídí napájení a zbylá dvě slouží k inverzi polarity.

Princip zařízení je jednoduchý.
Jakmile je nutné otevřít/zavřít okna, je nejdříve sepnuto relé, které řídí napájení.
V případě otevírání se dále sepnou obě relé řídící polaritu -- okna se začnou zavírat.
Pro dobu běhu motoru je stanoven fixní časový limit, který závisí na délce aktuátorů a jejich rychlosti. Po jeho uběhnutí se napájecí relé opět rozepne a aktuátory tedy přestanou být napájeny.
Zároveň se rozepne i relé řídící polaritu.
Jakmile chceme okna zavřít, relé řídící polaritu zůstanou rozepnuta a sepne jen relé ovládající napájení.



%\fxnote*[author=MB]{Text "Relé 3" nejde přečíst}{obrázek}
\section{Aktuátory pro manipulaci s okny}
Pro otevírání a zavírání oken používám lineární aktuátory.
Jejich pracovní napětí je 12~V.
Samotné aktuátory mají zabudovaná koncová čidla, po dojezdu do koncové polohy se samy zastaví.
Směr jejich pohybu lze ovládat inverzí polarity, kterou vykonává můj invertor polarity umístěný v řídící jednotce.

\section{Senzory}
Ke sledování stavu prostředí ve skleníku používám několik senzorů.
\begin{itemize}
	\item DHT 11, případně DHT 22
	\item DS18B20
\end{itemize}

\subsection{DHT 11/22}
DHT 11/22 jsou čidla, která používám pro měření vlhkosti vzduchu.
Jsou určena pro napájení na 5~V stejnosměrného napětí.
Jsou rozmístěna na krajích skleníku, z naměřených hodnot je následně vytvořen průměr. 
Počet použitých čidel se liší podle velikosti skleníku, ovšem pro základní verzi ProtoPlantu je z důvodu nedostatku GPIO u ESP32 nastaven limit tři senzory. Dva vnitřní a~jeden venkovní.

\subsection{DS18B20}
Tyto senzory používám k měření teploty.
Jsou zapojena přes 1-Wire sběrnici, což umožňuje připojení maximálně 10 těchto senzorů k jednomu pinu.
Samotný senzor má 3 piny: jeden napájecí (VCC), druhý datový (DQ) a třetí pro společnou zem napájecí i datovou (GND).
1-Wire sběrnice dokáže pracovat ve dvou režimech napájení senzorů, klasickém třídrátovém, nebo v režimu napájení parazitního (s využitím jen dvou drátů).
Z těchto dvou režimů využívá ProtoPlant režim klasický, třídrátový režim.

Maximální počet těchto senzorů, který je možno připojit k systému ProtoPlant je omezen na 6.

\section{Ovládání oken}
Okna jsou ovládána lineárními elektro-mechanickými aktuátory.
Jejich provozní napětí je 12~V.
Díky využití stejnosměrných motorů je možno měnit směr jejich pohybu jednoduchým obrácením polarity, což zajišťuje invertor polarity popsaný výše.
Rychlost posuvu je fixní, doba vysunutí tedy závisí pouze na délce aktuátorů.

\section{Ventilátor}
Pro zlepšení regulace vzdušné vlhkosti a zlepšení cirkulace vzduchu využívám pro drobnou korekci teploty ventilátor připevněný na stěně skleníku.
Ten se spustí v době, kdy je vnitřní teplota velmi nízko nad nastavenou horní mezí.

\section{Zavlažování}
Zavlažování obstarává vysokotlaké čerpadlo vody, které nasává vodu z nádrže umístěné přímo ve skleníku.
Je napájeno 12~V stejnosměrného napětí.
Čerpadlo je řízeno s pomocí SSR relé umístěného v řídící jednotce.
Voda se čerpá do rozvodů umístěných v horní části skleníku.
Na těchto rozvodech jsou nainstalovány rozprašovače.



%\fxnote*[author=MB]{Na obrázku je použito 220V místo 230V}{obrázek}
