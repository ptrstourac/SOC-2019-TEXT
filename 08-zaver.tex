\chapter*{Závěr}
\addcontentsline{toc}{chapter}{Závěr}

Mojí hlavní prioritou při tvorbě této práce byla snaha pomoct lidem, kteří vlastní skleník, ale nemají dostatek času se o něj starat.
V momentálním stavu je systém schopen automaticky otevírat okna, provádět zavlažování, měřit teplotu a vlhkost.
Možnosti využití jsou široké, vzhledem k tomu, že je systém univerzální a použitelný pro široký okruh rostlin, ať již okrasných, či užitkových.
Věřím, že právě díky univerzálnosti celého systému a poměrně nízké ceně si své využití najde především u široké veřejnosti.

Dále bych rád ProtoPlant rozvíjel po stránce možnosti skleník vzdáleně ovládat, případně kontrolovat přes internet například z dovolené.
V Internetu věcí vidím velké možnosti dalšího rozvoje.
Zároveň bych rád vylepšil i software skleníku a implementoval další funkce včetně možnosti záznamu změn teplot a generování grafů z těchto hodnot.

Co se týče cílů, o těch mohu říct, že jsem je splnil.
Podařilo se mi vytvořit systém pro automatizaci skleníku, který je univerzální, modulární, samostatný, jednoduše se instaluje a ovládá a je cenově dostupný.

Novinky z vývoje můžete sledovat na webových stránkách \\ www.protoplant.stpindustries.cz.

% \color{mygreen}
%[MB:] Vím, že Vás vůbec nešetřím... 
%\begin{itemize}
%	\item  Bude příloha se zdrojovými kódy vašeho software? Nebo alespoň vývojový a nebo stavový  diagram? 
%	\item  Jsme v kategorii elekro: bude nějaké schéma systému v Eaglu a pod.? 
%	\item  V přílohách by určitě měly být fotografie jak z velkého skleníku, tak modelu. 
%	\item  Chybí použitá literatura a odkazy, včetně datasheetů použitých modulů, srovnání s konkurencí, a pod. - probereme osobně/telefonicky 
	
%\end{itemize}
%\color{black}

