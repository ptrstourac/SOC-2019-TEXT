\chapter*{Úvod}
\addcontentsline{toc}{chapter}{Úvod} % přidá položku úvod do obsahu

Pěstování skleníkových rostlin, ať již okrasných, či užitkových, je velice roz\-ší\-řenou činností, pokud o ní mluvíme jako o činnosti zájmové.
Ovšem tato činnost je i velice časově náročná.
Lidé, kteří běžně chodí do práce mají stále méně času na péči o svoje skleníkové rostliny.
Mojí snahou je těmto lidem péči o rostliny částečně usnadnit zautomatizováním každodenních činností, jako je zalévání nebo větrání.

Mým řešením je systém na automatizaci skleníku s názvem ProtoPlant \cite{protoPlantWeb}.
Celý systém je navržen tak, aby byl jednoduchý na obsluhu, modulární a zároveň levný.
Po instalaci jej uživatel spustí, provede prvotní nastavení a o zbytek se již ProtoPlant postará sám.

Nabídka komerčních řešení je v tomto ohledu velmi chudá.
Samořejmě, existují průmyslová řešení, ovšem ta jsou určena pro firmy, nikoliv pro jednotlivce, jsou tedy velmi drahá.
Na webu existují návody na automatizaci skleníku, ovšem ty jsou určeny pro osoby, které již mají s elektrotechnikou nějaké zkušenosti, opět tedy jen pro užší skupinu lidí.

Při návrhu ProtoPlantu jsem dbal na to, aby měl několik důležitých vlastností:
\begin{itemize}
    \item jednoduchost používání a obsluhy -- chci, aby byl systém jednoduchý svojí obsluhou
    \item modularita -- celý systém je složen z modulů, které se dají jednoduše spojovat
    \item nízká cena -- chci, aby systém byl dostupný, jak cenově, tak i materiálně
    \item univerzálnost -- systém použitelný pro širokou škálu rostlin a plodin
    \item samostatnost -- systém se dokáže o sebe postarat sám po určitou dobu
    \item jednoduchá instalace -- uživatel dokáže ProtoPlant nainstalovat sám
\end{itemize}

Tyto vlastnosti jej činí ideálním pro použití v běžné domácnosti.

\newpage
