\chapter{Srovnání s konkurencí}
Co se týče konkurence ProtoPlantu, řadí se do dvou kategorií:
\begin{itemize} 
    \item průmyslová řešení
    \item řešení vytvořená kutily
\end{itemize}

\section{Průmyslová řešení}
V případě průmyslových řešení existují různé možnosti.
Ovšem prakticky všechny z nich jsou určeny pro využití ve velkých sklenících používaných ve velkozemědělství.
Zároveň je valná většina z nich instalována na míru -- není zde tedy možnost toho, že by si tento systém uživatel nainstaloval sám.

Co se ceny týče, ta se u~těchto systémů šplhá až do stovek tisíc, chybí zde tedy i cenová dostupnost těchto systémů.

Samotné řízení těchto průmyslových systémů je v drtivé většině případů zajišťováno tzv. PLC -- programovatelnými automaty.
Samotné PLC jsou velmi drahá a fungují jen čistě jakožto řídící hardware -- senzory a periferie tedy nejsou do ceny započítány.

Několik příkladů používaných PLC v tabulce \ref{table:plc_pricing}.
\begin{table}
    \centering
    \caption{Srovnání PLC a ProtoPLant řídící jednotky}
    \begin{tabular}{lllll}
    PLC        & Cena     & Výrobce       & Síťové vybavení & Příslušenství 
    (senzory, atp.)     \\
    \hline
    MPC 410\cite{MicropelPLC}    & 7 300,- & Micropel      & není            & \multirow{4}{*}{Nutné dokoupit }  \\
    S7 - 1200\cite{S7-1200PLC}   & 7 825,-   & Siement       & Ethernet        &                                   \\
    FX3s\cite{FX3sPLC}           & 6 000,-   & Mitsubishi    & Ethernet        &                                   \\
    Micro 810\cite{Micro810PLC}  & 2 693,-   & Allen Bradley & není            &                                   \\
    \hline
    ProtoPlant                   & 2 000,-   &               & WiFi, bluetooth & Senzory, napájení v ceně         
    \end{tabular}
    \label{table:plc_pricing}
    \end{table}

U dalšího hardwaru to je s cenou velmi podobné.
Firmy navíc dělají tyto instalace na zakázku, tudíž je třeba si připlatit i za práci.

\section{Kutilská řešení}
U~této kategorie sice konkurence existuje, ovšem projekty, které jsem našel na webu postrádají několik vlastností ProtoPlantu.

Tyto vlastnosti jsou:
\begin{itemize}
    \item jednoduchost -- k výrobě jsou třeba zkušenosti s elektrotechnikou
    \item univerzálnost -- většina těchto řešení je určena pro malé skleníky
    \item modularita
\end{itemize}

Daly by se sice nazvat cenově příznivé, ovšem co se týče dostupnosti, jsou tyto projekty vhodné pouze pro uživatele, kteří mají určité znalosti s~programováním a~elektrotechnikou.

Pro srovnání uvedu dva příklady.
\subsection{Arduino: automatický skleník}
Autor této bakalářské práce \cite{sklenikVavra} se přede mnou již pokusil vytvořit systém podobný ProtoPlantu založený na Arduinu.
Jde ovšem pouze o teoretickou práci, jeho systém nikdy nebyl uveden do praxe.

\subsection{Raspberry Pi Greenhouse}
Tento projekt \cite{raspPIgreenhouse} je založen na Raspberry PI.
Nutností pro jeho sestavení jsou určité znalosti v elektrotechnice a programování.

\begin{table}[]
    \centering
    \caption{Srovnání ProtoPlantu a projektu s RPI}
    \begin{tabular}{lll}
    Vlastnost                                                            & ProtoPlant                  & Projekt s Raspberry PI \\ \hline
    \multicolumn{1}{l|}{Modularita}                                      & \multicolumn{1}{l|}{\cmark} & \xmark                 \\
    \multicolumn{1}{l|}{Univerzálnost}                                   & \multicolumn{1}{l|}{\cmark} & \xmark                 \\
    \multicolumn{1}{l|}{Jednoduchost instalace}                          & \multicolumn{1}{l|}{\cmark} & \xmark                 \\
    \multicolumn{1}{l|}{Uvedení do praktického provozu}                  & \multicolumn{1}{l|}{\cmark} & \cmark                 \\
    \multicolumn{1}{l|}{Možnost instalace do různých velikostí skleníku} & \multicolumn{1}{l|}{\cmark} & \xmark                 \\
    \multicolumn{1}{l|}{Odolnost (vodotěsnost)}                          & \multicolumn{1}{l|}{\cmark} & \xmark                 \\
    \multicolumn{1}{l|}{Cena}                                            & \multicolumn{1}{l|}{\cmark} & \cmark                
    \label{table:comparationPPandRPI}
    \end{tabular}
    \end{table}

\section{Shrnutí}
Konkurence v~tomto oboru je skutečně chudá.
Na jedné straně jsou drahá průmyslová řešení instalovaná na míru, na straně druhé jsou tu projekty pro kutily, které ovšem jsou určeny jen pro uživatele se znalostí elektrotechniky a~programování.
Tuto díru na trhu se snažím vyplnit vytvořením ProtoPlantu -- systému dostupnému pro běžné uživatele a~domácnosti, který bude dostupný i~cenově i~materiálně.
