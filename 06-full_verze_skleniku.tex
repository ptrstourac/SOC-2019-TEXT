\chapter{Velká verze skleníku}
Celý systém testuji ve skleníku mého otce, ve kterém pěstuje orchideje.
V celém skleníku jsou nainstalovány rozvody vody i elektrického proudu.
Těsně pod stropem skleníku jsou rozvedeny trubice, na kterých jsou nainstalovány rozprašovače, slouží tedy k zavlažování.
Pro ovládání oken jsou zde nainstalovány lineární aktuátory popsané v sekci 1.2.2 s délkou výsuvu 45 cm.
Samotný skleník je umístěn z části pod zemí, z důvodu snížení nákladů na vytápění.
Pod úrovní podlahy je umístěna nádrž na vodu s kapacitou 3,5~m$^3$ pro použití při zavlažování.
Do této nádrže je sbírána dešťová voda z okapů.

Co se týče senzorů, v této plné verzi jsou instalovány 3 senzory DHT i 3 DS18B20.
Ty jsou rovnoměrně rozmístěny tak, aby bylo pokrytí jejich měření rozloženo po celém skleníku.
Jsou připevněna na kabelech visících ze stropu skleníku.

Verze systému 1.0 byla dokončena teprve nedávno, připravuji tedy nasazení pro testovací provoz v tomto skleníku.
Co se týče mých očekávání, předpokládám, že výsledkem testu bude značná úspora práce.
Dále očekávám úsporu tepla i~energií, vzhledem k tomu, že se bude větrat pouze, pokud to bude nutné.




 %\color{mygreen}
%[MB:] Chybí údaje o testovacím provozu: 
%\begin{itemize}
	%\item jak dlouho testování probíhá ?
	%\item s jakými výsledky (úspora tepla, vody, práce ... ) jak velká?
	%\item kolik je jakých čidel a kde jsou osazená? 
	%\item  jaká byla spolehlivost provozu? 
	%\item  Jaké byly náklady pro uvedení do provozu? 
	%\item  Jak velká je nádrž a odkud se bere voda? Je přívod taky automatický? 
	
%\end{itemize}
 %\color{black}
