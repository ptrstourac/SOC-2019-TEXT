\chapter{Software a popis funkce}
Tato kapitola se týká popisu funkce celého systému ProtoPlant a softwarové stránky projektu.

\section{ProtoPlant knihovna}
Softwarová knihovna PPlant vytvořená mnou obsahuje jednodušší podpůrné funkce ovládající hardware skleníku.
Mezi ně patří:
\begin{itemize}
    \item Spouštění a zastavování čerpadla vody
    \item Práce s LCD
\end{itemize}

\section{Start-up sekvence}
Tato sekvence se sestává z několika operací nutných pro správnou inicializaci systému.

Celá sekvence by se dala rozdělit na několik úkonů:
\begin{enumerate}
    \item Inicializace LCD
    \item Kontrola systému
    \item Inicializace senzorů
    \item Nastavení krajních hodnot parametrů prostředí, nebo jejich načtení z paměti
\end{enumerate}
Po dokončení celé sekvence se systém automaticky přepne do režimu standby, ve kterém na pozadí monitoruje vlhkost a teplotu.

\subsection{První spuštění}
Při prvním spuštění se systém při provádění startup sekvence zastaví v bodě 4, při kterém musí uživatel nastavit krajní hodnoty vlhkosti a teploty.
Ty uloží následně do flash paměti čipu ESP32, kde zůstanou zachované i v případě, že se systém vypne, nebo vypadne proud.
Po zadání a~uložení hodnot systém následně pokračuje v běhu přechodem do režimu standby.

\subsection{Opakované spuštění}
Při opakovaném spuštění již systém v bodě 4 nastavení parametrů nepožaduje, místo toho si načte uložené hodnoty z paměti a pokračuje v normálním chodu.

\section{Režim STANDBY}
Po proběhnutí startup sekvence se systém automaticky přepne do režimu STANDBY.
V tomto režimu systém na pozadí periodicky měří hodnoty vlhkosti a teploty, které následně porovnává s~hodnotami uloženými ve flash paměti.
Zároveň naměřené hodnoty vypíše na LCD~displej.

\section{Zavlažování a měření vlhkosti}
Vlhkost vzduchu je pravidelně měřena třemi senzory DHT22 rozmístěnými po skleníku. %\fxnote*[author=MB]{Výše píšete o maximálně dvou}{několika}
Interval pro měření teploty je nastaven na 1 minutu, což je optimální doba vypozorovaná z testování, které jsem dříve provedl.
Z dat naměřených jednotlivými senzory je následně vypočten aritmetický průměr.
Výsledná hodnota je porovnaná s krajními mezemi vlhkosti nastavenými uživatelem.
Pokud je výsledek v mezích, skleník pokračuje v~periodickém měření vlhkosti. Pokud je vlhkost příliš nízká, systém sepne solid state relé umístěné v řídící jednotce. 
Čerpadlo, které je k němu připojené se tedy spustí. 
%\fxnote*[author=MB]{A co když je ve skleníku moc vlhko?}{v mezích},
V případě, že je vlhkost příliš vysoká, systém otevře okna, případně spustí ventilátor umístěný na stěně skleníku.

\subsection{Softwarová stránka zavlažování}
Zavlažování je po softwarové stránce velmi jednoduchým úkonem.
Měření a zpracování teploty zajišťují knihovny DHT a DHT\_U přímo od výrobce senzorů DHT22, Adafruitu.
Samotné porovnání teploty s mezní hodnotou je provedeno v hlavním programu main. 
Spouštění čerpadla se následně provede zavoláním příkazu pro spuštění čerpadla, který obsaženého v knihovně PPlant.

\section{Ovládání oken}
Systém v pravidelných intervalech měří teplotu.
Tuto naměřenou hodnotu porovnanává s mezními hodnotami uloženými v paměti.
V případě, že je teplota v mezích, systém pokračuje v normálním chodu.
Pokud je teplota příliš vysoká, okna se otevřou, pokud je teplota moc nízká, zavřou se.

\subsection{Software pro ovládání oken}
Pro řízení inventoru polarity aktuátorů ovládajících okna jsem napsal vlastní knihovnu.
% Celá je řízena těmito pěti  \fxnote*[author=MB]{Více v poznámce ve zdrojovém textu }{funkcemi.}
\begin{verbatim}  Mezi některé funkce této knihovny patří:
    window.begin();         
    window.getstate();      
    window.close();         
    window.open();          
    window.changestate();
\end{verbatim}

%pro formátování kódu doporučuju balíček listings, viz https://en.wikibooks.org/wiki/LaTeX/Source_Code_Listings
% například: \lstinputlisting{priklady_c/blikani_LED1.cpp}
% a nebo alespoň písmo verbatim \verb@"psací stroj"@

Nejdříve se knihovna inicializuje zavoláním funkce window.begin();.
Funkce getstate(); je využívána pro zjištění, zda jsou okna otevřená, či nikoliv. 
Jejím výstupem je logická hodnota true/false.
Funkce open(); a close(); jsou využity, jak již jejich název napovídá k otevření, či zavření oken.
S pomocí funkce changestate(); jsme schopni přepínat mezi otevřeným/zavřeným stavem nezávisle na tom, jakém stavu jsou před jejím zavoláním.
V případě, že jsou okna před jejím zavoláním otevřená, zavolání funkce je zavře a naopak.

